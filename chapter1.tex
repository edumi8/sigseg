\section{Introduction}
GreenGrid Energy is a large-sized enterprise \cite{Portugal2020} operating in the renewable energy sector, with a particular focus on smart grid technology. The organization employs approximately 400 professionals and reported revenue of roughly 35 million euros in the most recent fiscal year. Its business model revolves around three key revenue streams:
\begin{enumerate}
    \item Smart grid infrastructure deployment (45\%),
    \item Renewable energy supply contracts (40\%),
    \item Energy monitoring and analytics software (15\%).
\end{enumerate}

With the growing adoption of renewable energy solutions and \ac{IoT} devices in power grids, GreenGrid Energy finds itself navigating a complex landscape that blends \ac{OT} and \ac{IT}. This convergence elevates the importance of robust cybersecurity measures and well-defined governance strategies.

\section{Governance and Leadership}
GreenGrid Energy operates under a governance model designed to balance strategic oversight with operational efficiency:
\begin{itemize}
    \item \textbf{Board of Directors}: Shapes the strategic vision, approves major investments, and ensures compliance with energy regulations.
    \item \textbf{\ac{CEO}}: Oversees day-to-day leadership and reports directly to the Board.
    \item \textbf{Executive Management Team}:
        \begin{itemize}
            \item \emph{Chief Financial Officer (CFO)}: Oversees budget planning, financial compliance, and resource allocation.
            \item \emph{Chief Technology Officer (CTO)}: Guides the technological roadmap, including research and product development.
            \item \emph{Chief Operating Officer (COO)}: Manages operational functions, including engineering, supply chain, and service delivery.
            \item CIO
            \item \emph{Chief Information Security Officer (CISO)}: Establishes cybersecurity strategies and ensures best practices across OT and IT domains.
        \end{itemize}
    \item \textbf{Department Directors or Managers}: Lead their respective divisions and report to the relevant C-level executive.
\end{itemize}



\section{Key Departments}
Although the company maintains several functional areas, the following six departments represent the backbone of GreenGrid Energy:

\subsection{Engineering \& Grid Operations}
This department focuses on designing, maintaining, and optimizing the physical power distribution network. Responsibilities include real-time supervision of Supervisory Control and Data Acquisition (SCADA) systems, ensuring grid reliability, and implementing upgrades to maintain efficiency. Collaboration with R\&D is critical when new technologies are tested or introduced into the grid environment.

\subsection{Research \& Development (R\&D)}
The R\&D department concentrates on innovation in predictive analytics, grid optimization algorithms, and the development of cutting-edge solutions for renewable energy systems. Protecting intellectual property and ensuring secure, controlled testing environments are essential concerns given the competitive nature of the energy technology market.

\subsection{Customer Services \& Billing}
All client-facing activities fall under this area, including account management, billing, and technical support for both residential and enterprise customers. Confidentiality and compliance with data protection regulations are critical due to the handling of personal and financial information.

\subsection{Supply Chain Management}
This division is responsible for procuring hardware and services crucial to the operation of the power grid infrastructure. Maintaining resilient supplier relationships, vetting vendor security practices, and ensuring timely deliveries are among its primary tasks. The security of hardware components entering the network is an increasing concern, given the potential for compromised devices.

\subsection{Human Resources (HR)}
Recruitment, training, performance evaluation, and employee welfare fall under HR. As part of the broader organizational security posture, HR manages background checks, security awareness training, and role-based access control for employees.

\subsection{Legal \& Administration}
This area ensures that GreenGrid Energy remains compliant with ever-evolving industry standards and regulations. Key tasks include drafting and reviewing contracts, monitoring regulatory changes in the energy sector, and overseeing budget adherence within administrative activities.

\section{Infrastructure and Security Context}

\subsection{Core Technologies and Systems}
GreenGrid Energy relies on both operational and corporate infrastructure:
\begin{itemize}
    \item \textbf{SCADA Systems}: Facilitate real-time monitoring and control of power grid assets.
    \item \textbf{Hybrid Cloud (AWS/Azure) and Edge Computing}: Provides a scalable framework for advanced data analytics and IoT management.
    \item \textbf{Enterprise Solutions}: Microsoft 365 for collaboration and SAP ERP for resource planning.
\end{itemize}

\subsection{Network Topology}
The organization maintains two distinct networks:
\begin{itemize}
    \item \textbf{Operational Technology (OT) Network}: Houses SCADA systems, real-time control devices, and IoT sensors responsible for the grid's operation.
    \item \textbf{Information Technology (IT) Network}: Hosts corporate functions, including email servers, ERP, and HR systems.
\end{itemize}
Secure segmentation and controlled data sharing between these two networks is vital to protect sensitive operations from potential external threats.

\subsection{Threat Landscape}
In the current geopolitical and technological context, GreenGrid Energy faces the following threats:
\begin{itemize}
    \item \textbf{Nation-State Actors}: Motivated by political or strategic objectives, these attackers may target critical infrastructure to disrupt services or exfiltrate data.
    \item \textbf{Advanced Persistent Threats (APTs)}: Exploit IoT and smart meter vulnerabilities to infiltrate the network, often remaining undetected for prolonged periods.
    \item \textbf{Supply Chain Risks}: Compromised components can introduce hidden backdoors or vulnerabilities, emphasizing the importance of secure procurement practices.
\end{itemize}